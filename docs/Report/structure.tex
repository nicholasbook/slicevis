\chapter{Package Structure}
This chapter details the structure of the \emph{slicevis} package repository which is publicly available on GitLab \cite{repo}.

Besides the Python source code for the package, located in the \emph{slicevis} sub-directory, the repository contains a \texttt{README}, a license file (\texttt{LICENSE}), and build instructions for pip. Additionally, there is an directory for examples and one for documentation. 

The README is formatted using Markdown and it serves as a short description and usage guide for the package. 

\emph{Slicevis} is licensed under the MIT License, which permits free use, modification, and redistribution as long as credit is given \cite{mit}. 

The build instructions are specified in the files \texttt{pyproject.toml},\texttt{ setup.cfg}, and \texttt{setup.py} according to the Python Packaging User Guide \cite{pypa}. In \texttt{pyproject.toml}, the build tool is configured to Setuptools. All necessary package metadata and dependencies are specified in \texttt{setup.cfg}. The dependencies are numpy, gffio, nibabel, matplotlib, plotly, ipywidgets, nbformat, wheel, and pandas. A \texttt{setup.py} file is included for backwards-compatibility reasons, as described here \cite{legacy_config}.

The documentation directory includes the final presentation and this report as a PDF plus all required \LaTeX files to make further changes to the report.

The examples directory contains several datasets and a Jupyter notebook that demonstrates how to use the package. For the datasets I chose a CT scan of a mouse, including a multi-organ segmentation, a CT scan of a human upper body with labeled lung tumors, and a prediction of a neural network for the same lung scan. More details are given in the following chapter titled Use-Cases. A special license file is also included which details the origins and copyright notices for all examples.